\documentclass{article}
\usepackage[utf8]{inputenc}
\usepackage[spanish]{babel}
\setlength\parindent{0pt}
\setlength{\parskip}{\baselineskip}%
\newcommand{\HRule}{\rule{\linewidth}{0.5mm}}
\usepackage{color}
\usepackage{graphicx}

\begin{document}
    \begin{center}
    	\includegraphics[scale=0.75]{logoUSB}\\[0.1cm]
    	\textsc{Universidad Simón Bolívar}\\[0.7cm]
        { \LARGE \bfseries Práctica 1}\\[0.3cm]
        \textsc{Gustavo Gutiérrez - 11-10428}\\[0.1cm]
    \end{center}
\clearpage

\section{Pregunta 1}
\begin{enumerate}
	\item gpg --gen-key prueba
	\item El passphrase se utiliza como contraseña de acceso al archivo de la clave
		privada para evitar que terceros puedan usar nuestra clave privada. En el caso de que nos vulneren la llave,
		el vencimiento de la misma garantiza que el tercero malicioso no pueda usar la llave indefinidamente.
	\item \includegraphics[scale=0.02]{captura}\\ Gpg muestra la siguiente información:
	\begin{itemize}
		\item La clave publica bajo la etiqueta pub.
		\item El fingerprint de la clave privada. Es una versión abreviada de la misma.
		\item La información del usuario asociado a la llave.
		\item La subclave generada para encriptar.
	\end{itemize}
\end{enumerate}
\end{document}

